
\begin{longtable}{|p{4cm}|p{4cm}|p{4cm}|}
  \hline
  \textbf{Termine} & \textbf{Descrizione} & \textbf{Sinonimi} \\
  \hline
  \endfirsthead
  \hline
  \textbf{Termine} & \textbf{Descrizione} & \textbf{Sinonimi} \\
  \hline
  \endhead
  \hline
  \endfoot
  
  \hline
  Attori & & \\
  \hline
  Chef & Stabilisce il menù per gli eventi e ne supervisiona la preparazione & \\
  \hline
  Cuoco & esegue le ricette & \\
  \hline
  Personale di servizio & Coloro che si occupano del servizio durante l'evento stesso & Camerieri, Lavapiatti, Sommelier \\
  \hline
  Utente & Colui che richiede l'organizzazione dell'evento & \\
  \hline
  Cuoco esperto & Cuoco con una certa esperienza tale da svolgere compiti complessi o sapere gestire situazioni di imprevisti & \\
  \hline
  Organizzatore & La persona che gestisce il personale di servizio, gli chef e gli eventi & \\
  \hline
  Dominio delle applicazioni & & \\
  \hline
  Evento & Contesto in cui viene fornito il servizio di catering. Può essere ad es. un pranzo, una cena, un aperitivo, un buffet, una festa di laurea. Prevede due momenti diversi, il lavoro in cucina e il servizio. Se ne fa carico un organizzatore, mentre la cucina è affidata a un chef. Un evento può essere semplice, e prevedere un singolo servizio, o complesso. Un evento complesso può prevedere più servizi (pranzo e cena, colazione e pranzo, coffee-break mattino e pomeriggio, ecc) in un'unica giornata e/o prevedere più giornate. Gli eventi possono inoltre essere classificati come ricorrenti nel caso in cui si ripetano con una certa regolarità. & \\
  \hline
  Emergenze & Imprevisti che influenzano sull'"Organizzazione dell'evento ,come può essere un malfunzionamento in cucina ,queste problematiche influenzeranno la gestione dell'evento & \\
  \hline
  Disponibilità & Predisposizione allo svolgere un'attività lavorativa in un determinato orario & \\
  \hline
  Evento ricorrente & Servizio catering che si ripete periodicamente & \\
  \hline
  Schema & Impostazione sommaria sul come svolgere un evento & \\
  \hline
  Tabellone & Strumento disponibile a tutti gli addetti alla cucina e al servizio suddiviso nelle seguenti parti: \begin{itemize} \item Suddivisione dei turni per svolgere i compiti \item Assegnamento del compito da svolgere \item Note informative \end{itemize} & \\
  \hline
  Compito & Insieme di azioni necessarie affinché una ricetta o una preparazione o un elemento necessario al servizio sia compiuto & \\
  \hline
  Ricettario & insieme di ricette & \\
  \hline
  Ricetta & Indicazione degli ingredienti, delle modalità di preparazione di un cibo & \\
  \hline
  Turno & Periodo tra due intervalli in cui si deve svolgere un'attività specifica & Turno standard \\
  \hline
  Turno ricorrente & Turno che si ripete con cadenza periodica & \\
  \hline
  Preparazione & Simile a una ricetta, ma descrive come realizzare un preparato da utilizzare in un'altra. & \\
  \hline
  Menù & Descrive quali piatti vengono proposti in un dato servizio (nell'ambito di un evento) Si compone di diverse voci, opzionalmente organizzate in sezioni. & \\
  \hline
  Cibo & Piatti preparati (seguendo una ricetta del ricettario) per essere consumati durante un servizio & \\
  \hline
  Foglio riepilogativo & Foglio riassuntivo posseduto da Chef in cui sono inserite le informazioni sull'evento i turni per lo svolgimento dei compiti ,il menù con le ricette il tipo di preparazione da effettuare e le relative quantità & \\
  \hline
  Turni in blocco & Insieme di turni suddivisi per mensilità o tipologia di compiti da svolgere o per ricorrenza & \\
  \hline
  Servizio & Attività che consiste nel servire ai partecipanti i piatti e le bevande preparati. Include l'allestimento preliminare e il rigoverno al termine. Può andare dal semplice buffet all'allestimento di una sala ristorante. Uno stesso evento può prevedere più servizi, anche in sedi diverse. Ciascun servizio avrà una precisa fascia oraria, e naturalmente un proprio menu e un proprio staff di supporto. & \\
  \hline
  Calendario dei turni & Insieme di turni necessari per svolgere i compiti suddivisi in blocchi & \\
  \hline
  Ricettario & Insieme di ricette & \\
  \hline
  Porzioni & Quantità indicativa di un piatto finito per una persona & \\
  \hline
  Cucina & Ambiente riservato e attrezzato per la preparazione e la cottura dei cibi necessari per l'evento & \\
  \hline
  Preparazione in sede & Fase in cui i cuochi si dedicano ad anticipare alcune preparazioni e ricette necessarie per un servizio. & \\
  \hline
  Turni consecutivi & Un turno seguito da un altro turno & \\
  \hline
  Disponibilità & Personale libero a svolgere uno specifico compito in uno specifico turno & \\
  \hline
  Sede & Ubicazione della società di catering e delle relative cucine & \\
  \hline
  \end{longtable}