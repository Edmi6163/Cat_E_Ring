\section*{Contratti per UC "Gestire i compiti della cucina"}

\subsection*{Precondizione generale}
\begin{itemize}
    \item L'utente deve essere identificato come un'istanza \textit{ch} di Chef.
\end{itemize}

\subsection*{1. creaFoglioRiepilogativo(titolo?:testo, evento:Evento)}

\subsubsection*{Precondizioni:}
\begin{itemize}
    \item \textit{ch} deve essere incaricato per questo evento.
\end{itemize}

\subsubsection*{Postcondizioni:}
\begin{itemize}
    \item Se l'evento è assegnato a \textit{ch} e non esiste un'istanza di foglio riepilogativo \textit{fr}, viene creata una nuova istanza \textit{fr} di tipo \textit{FoglioRiepilogativo}.
    \item Se è specificato un titolo, \textit{fr.titolo = titolo}.
    \item \textit{ch} è proprietario di \textit{fr}.
\end{itemize}

\subsection*{1a.1 scegliFoglioRiepilogativo(foglioRiepilogativo:FoglioRiepilogativo)}

\subsubsection*{Precondizione:}
\begin{itemize}
    \item L'evento a cui si riferisce il foglioRiepilogativo prevede che \textit{ch} sia proprietario.
\end{itemize}

\subsubsection*{Postcondizione:}
\begin{itemize}
    \item Il foglioRiepilogativo selezionato diventa il documento attivo per ulteriori operazioni.
\end{itemize}

\subsection*{1b.1 modificaTitoloFoglioRiepilogativo(nuovoTitolo:testo)}

\subsubsection*{Precondizione:}
\begin{itemize}
    \item Deve già esistere un'istanza di foglio riepilogativo \textit{fr}.
    \item Se viene specificato il titolo, allora \textit{fr.titolo = titolo}.
    \item Il nuovo titolo non deve essere già presente su un altro foglio riepilogativo.
\end{itemize}

\subsubsection*{Postcondizione:}
\begin{itemize}
    \item Il nuovo titolo per il foglio riepilogativo \textit{fr} è \textit{fr.titolo = nuovoTitolo}.
\end{itemize}

\subsection*{1c.1 modificaContenutoFoglioRiepilogativo(foglioRiepilogativo:FoglioRiepilogativo)}

\subsubsection*{Precondizione:}
\begin{itemize}
    \item È stata modificata un'istanza \textit{fr} di foglio riepilogativo.
\end{itemize}

\subsubsection*{Postcondizione:}
\begin{itemize}
    \item È stata modificata un'istanza \textit{fr} di foglio riepilogativo.
    \item \textit{fr.titolo = foglioRiepilogativo.titolo}.
    \item \textit{fr.turniCucina = foglioRiepilogativo.turniCucina}.
    \item \textit{fr.ricetta = foglioRiepilogativo.ricetta}.
    \item \textit{fr.preparazioneAvanzata = foglioRiepilogativo.preparazioneAvanzata}.
    \item \textit{fr.quantitàPerPreparazioneAvanzata = foglioRiepilogativo.quantitàPerPreparazioneAvanzata}.
\end{itemize}

\subsection*{1d.1 copiaFoglioRiepilogativo(foglioRiepilogativo:FoglioRiepilogativo)}

\subsubsection*{Precondizione:}
\begin{itemize}
    \item Deve esistere un'istanza \textit{fr} di foglio riepilogativo.
\end{itemize}

\subsubsection*{Postcondizione:}
\begin{itemize}
    \item È stata creata un'istanza \textit{fr} di foglio riepilogativo.
    \item \textit{fr.evento = foglioRiepilogativo.evento}.
    \item \textit{fr.titolo = foglioRiepilogativo.titolo}.
    \item \textit{fr.turno = foglioRiepilogativo.turno}.
    \item \textit{fr.menù = foglioRiepilogativo.menù}.
    \item \textit{fr.ricetta = foglioRiepilogativo.ricetta}.
    \item \textit{fr.preparazioneAvanzata = foglioRiepilogativo.preparazioneAvanzata}.
    \item \textit{fr.quantitàPerPreparazioneAvanzata = foglioRiepilogativo.quantitàPerPreparazioneAvanzata}.
\end{itemize}

\subsection*{1e.1 cancellaFoglioRiepilogativo(foglioRiepilogativo:FoglioRiepilogativo)}

\subsubsection*{Precondizione:}
\begin{itemize}
    \item \textit{ch} è proprietario di \textit{foglioRiepilogativo}.
    \item \textit{foglioRiepilogativo} non è in uso in alcun Evento.
\end{itemize}

\subsubsection*{Postcondizione:}
\begin{itemize}
    \item Ogni contenuto di \textit{foglioRiepilogativo} (menù, titolo, lista dei turni della cucina, ricetta) viene eliminato.
\end{itemize}

\subsection*{2. aggiungiCompito(compito:Compito)}

\subsubsection*{Precondizione:}
\begin{itemize}
    \item È in corso la definizione di un tabellone composto da istanze di \textit{Compito} \textit{c}.
\end{itemize}

\subsubsection*{Postcondizione:}
\begin{itemize}
    \item Compito aggiunto alla lista dei compiti.
    \item Compito aggiunto al foglio riepilogativo.
\end{itemize}

\subsection*{3. ordinaCompitiDifficolta(compito:Compito)}

\subsubsection*{Precondizione:}
\begin{itemize}
    \item È in corso la definizione di un tabellone composto da istanze di \textit{Compito} \textit{c}.
\end{itemize}

\subsubsection*{Postcondizione:}
\begin{itemize}
    \item I compiti sono ordinati in base alla difficoltà di esecuzione.
\end{itemize}

\subsection*{4. ordinaCompitiPriorita(compito:Compito)}

\subsubsection*{Precondizione:}
\begin{itemize}
    \item È in corso la definizione di un tabellone composto da istanze di \textit{Compito} \textit{c}.
\end{itemize}

\subsubsection*{Postcondizione:}
\begin{itemize}
    \item I compiti sono ordinati in base alla necessità temporale di completamento.
\end{itemize}

\subsection*{5. ordinaCompitiTempistiche(compito:Compito)}

\subsubsection*{Precondizione:}
\begin{itemize}
    \item È in corso la definizione di un tabellone composto da istanze di \textit{Compito} \textit{c}.
\end{itemize}

\subsubsection*{Postcondizione:}
\begin{itemize}
    \item Ordino compito in base alla lunghezza temporale che esso necessità per essere svolto.
\end{itemize}

\subsection*{6. piùFogliRiepilativi(foglioRiepilogativo:FoglioRiepilogativo, titolo?:Titolo)}

\subsubsection*{Precondizione:}
\begin{itemize}
    \item \textit{ch} è proprietario di questo evento.
    \item Esiste già un'istanza di \textit{foglioRiepilogativo} \textit{fr} per questo evento.
\end{itemize}

\subsubsection*{Postcondizione:}
\begin{itemize}
    \item Creo una nuova istanza \textit{fr2} di \textit{FoglioRiepilogativo}.
    \item [Se si specifica il titolo] \textit{fr2.titolo = titolo}.
    \item \textit{ch} è proprietario di \textit{fr2}.
\end{itemize}

\subsection*{7. consultaTabellone()}

\subsubsection*{Precondizione:}
\begin{itemize}
    \item 
\end{itemize}

\subsubsection*{Postcondizione:}
\begin{itemize}
    \item 
\end{itemize}

\subsection*{8. assegnaCompito(nome?:NomeDipendente, cognome?:CognomeDipendente, compito:Compito)}

\subsubsection*{Precondizione:}
\begin{itemize}
    \item È in corso l'assegnamento dei compiti da svolgere (non assegnato a nessun dipendente).
    \item Se il compito ha tempistiche troppo lunghe rispetto al turno di assegnamento, prolungo il compito al turno successivo disponibile.
\end{itemize}

\subsubsection*{Postcondizione:}
\begin{itemize}
    \item [Se si specifica un nome] \textit{compito.nomeAssegnato = nome}.
    \item [Se si specifica un cognome] \textit{compito.cognomeAssegnato = cognome}.
    \item \textit{compito.tipoCompito = tipoCompito}.
\end{itemize}

\subsection*{9. consultoListaTurniCucina()}

\subsubsection*{Precondizioni:}
\begin{itemize}
    \item 
\end{itemize}

\subsubsection*{Postcondizioni:}
\begin{itemize}
    \item 
\end{itemize}

\subsection*{10. consultoFoglioRiepilogativo(foglioRiepilogativo:FoglioRiepilogativo)}

\subsubsection*{Precondizioni:}
\begin{itemize}
    \item 
\end{itemize}

\subsubsection*{Postcondizioni:}
\begin{itemize}
    \item 
\end{itemize}