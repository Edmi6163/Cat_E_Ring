\section{User story}

\subsection{User story di Viola,organizattrice di eventi e wedding planner}
\begin{enumerate}
  \item Compila la richiesta di un nuovo evento con durata, numero di servizi richiesti, data, luogo, numero di partecipanti e alcune informazioni aggiuntive su tipologie di evento
  \item Decide quale chef si occuperà dell'evento 
  \item Quando l'evento si avvicina decide anche il personale per i vari servizi.
  \item In caso di personale in eccesso  viene liberato 
  \item Quando un evento termina segna delle note per eventi futuri, soprattutto in caso di eventi ricorrenti
  \item In caso di rimanenze vengono segnate agli chef
\end{enumerate}

\subsection{User story di PierPaolo, organizzatore di eventi e proprietario di una società di catering}
\begin{enumerate}
  \item  Quando arriva una richiesta per un evento per prima cosa verifica lo stato della cucina nei giorni precedenti, per accertarsi della disponibilità di questa 
  \item Sceglie 1 tra i 4 chef presenti nell'azienda in base alla specializzazione di questo/taglio che si vuole dare all'evento.
  \item Si consulta con lo chef di persona prima di prendere una qualsiasi decisione
  \item Dopo che lo chef ha scelto i menù richiede il personale 
  \item Si occupa dell'allestimento della sala e in caso venga richiesto anche il luogo
  \begin{enumerate}
  \item In caso venga annullato l'evento viene segnato che non avviene più in modo dal iberare il personale
 \item Se viene annullato a meno di una settimana viene fatta pagare una penale
  \end{enumerate}
\end{enumerate}

\subsection{Uc Dettagliato di Viola,organizattrice di eventi e wedding planner}
  % create a table with 2 columns "attore" and "sistema"

\begin{tabular}{|p{5cm}|p{5cm}|}
  \hline
  Attore & Sistema \\
  \hline
  Compila la richiesta di un nuovo evento & Registra il nuovo evento con dati rilevanti per questo \\
  \hline
  Decide quale chef si occuperà dell'evento & Assegna un chef all'evento \\ 
  \hline
  Decide quale chef si occuperà dell'evento & Assegna un chef all'evento \\
  \hline
  In possimità dell'evento decide il personale & Assegna il personale all'evento \\
  \hline
  In caso di personale in eccesso viene liberato & Libera il personale \\
  \hline
  Terminato un evento segna delle note per eventi futuri & Registra le note per ausilio degli eventi futuri\\
  \hline
\end{tabular}
\newline
\caption{Uc Dettagliato di Viola,organizattrice di eventi e wedding planner}
\newline
\begin{tabular}{|p{5cm}|p{5cm}|}
  \hline
  Attore & Sistema \\
  \hline
  Quando arriva una richiesta per prima cosa verifica lo stato della cucina & Verifica la disponibilità della cucina \\
  \hline
  Sceglie uno tra i 4 chef in base alla specializzazione di questo e al mood& Assegna un chef all'evento \\
  \hline
  Consulta lo chef di persona & Prende le decisioni in collaborazione con lo chef \\
  \hline
  Dopo che lo chef ha scelto i menù richiede il personale & Assegna il personale all'evento \\
  \hline
  Si occupa dell'allestimento della sala & Allestisce la sala \\
  \hline
  Terminato un evento 
\end{tabular}
\newline
\caption{Uc Dettagliato di PierPaolo, organizzatore di eventi e proprietario di una società di catering}
\newpage
% table of extension for tabular up
 \begin{tabular}{|p{5cm}|p{5cm}|}
  \hline 
  Attore & Sistema \\
  \hline
  In caso venga annullato l'evento viene segnato che non avviene più & Libera il personale se viene annullato l'evento \\
  \hline
  Se l'evento viene annullato a meno di una settimana viene fatta pagare una penale & Fissa una penale se l'evento viene annullato a meno di una settimana \\
 \end{tabular}
 \newline
\caption{Estensioni PierPaolo}