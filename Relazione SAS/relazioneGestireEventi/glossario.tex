\section{Glossario}
\begin{tabular}{|p{3cm}|p{5cm}|p{3cm}|}
  \hline
  Termine & Descrizione & Sinonimi \\
  \hline
  ATTORI & & \\
  \hline
  Chef & Stabilisce i men`u per gli eventi e superviisona la cucina & \\
  \hline
  Cuoco & Prepara il cibo & \\
  \hline
  Organizzatore & Gestisce il personale e gli eventi & \\
  \hline
  Personale & Può intendere sia tutti i dipendenti o solamente quelli soggetti a turno & cameriere,\newline lavapiatti,\newline sommelier \\
  \hline
  Personale di servizio & Si occupano del serivizio durante l'evento stesso & \\
  \hline
  Dominio dell'applicazione & & \\
  \hline
  Cibo & Pietanze preparate per essere consumate durante un servizio &Pietanza \\
  \hline
  Evento & Contesto in cui viene fornito il servizio di catering(ad esempio: pranzo,cena o aperitivo),l'evento può essere:  \newline \begin{itemize}
    \item in corso o terminato
    \item riorrente o definito da una scheda evento 
  \end{itemize} & \\ 
  \hline
  Menù & Descrive quali piatti vengono rpoposto in un dato servizio, è composto da diverse voci,a volte divise in diverse sezioni & \\
  \hline
  Ricetta & Descrivere come preparare un piatto da servire durante il servizio,è caratterizzato da un nome & \\
  \hline
  Ricettario & Raccolta di ricette & \\
  \hline
  Servizio & Fase di un evento in uci avviene effettuato il catering ai partecipanti & \\
  \hline
  Sezione di menù & Partizionamento (opzionale) di un menù [..] & \\%TODO finisci 
  \hline
  Voce di menù & Signolo elemento di un menù & \\
  \hline
  Preparazioni & descive come realizzare un preparato da utilizzare in un'altra & \\
  \hline
\end{tabular}

